%conclusion

%description of the project and theory.
%description of the project and theory.
Coupling the light from optical fiber to photonic waveguide (fiber-to-chip) is a common topic for research and application in optical communication. As the light source the normal fiber has a generally an interface bigger than the dimension of the photonic waveguide. In order to promote the coupling efficiency the tapered lensed fibers (TLF) are usually applied as the light source. Thus the coupling between TLF and photonic waveguide becomes an attractive agenda of the optic research. \\ 

%purpose of this work
This work aims for the optimal solution for the effective coupling between TLF and photonic waveguide. In order to achieve this goal the coupling models have been constructed and simulated with the aid of CST MWS. In this work the modeling procedure and the analyses of the result are recounted.\\

%the content and the result of this work
%summary of each chapter, 
In chapter. \ref{chp:background} the basic knowledge about the geometric optic, fiber optic, Gaussian beam, finite integration technigue and S-Parameter is listed. The above knowledge could be helpful to understand some terms of this work. The chapter. \ref{chp:model} gives readers of this work at first an impression about the technical details of the experimental objects. Then the modeling procedure, how the model is simplified and how the properties of models look like, is introduced. Especially, two types of TLF models are compared and one is finally chosen for the further discussion. In chapter. \ref{chp:optim} simulations about the effective coupling between TLF and the waveguide is divided into five parts. In the first part, the simulation aims at the effect of displacing the waveguide on the coupling efficiency. In the second part we try the same coupling configuration in oil environment instead of in air. In the third part the effect of the refractive index is discussed. After that, the fourth and fifth parts provide two important techniques of waveguide interface, tapered and lensed interface, for promoting the coupling ability.\\        
 
%compare and conclude the results, advice for experiment 
According the results from all simulations in this work, a good designed waveguide interface can greatly affect the coupling ability of Fiber-to-Chip. The original coupling arrangement in this work achieved an efficiency $48.9\%$. The waveguide with $n=1.8$ in coupling leads to an attractive result $62.5\%$. The simply constructed tapered interface gained maximally a value about $54\%$. In comparison to the tapered interface, lensed interface of the waveguide can catch the efficiency about $69\%$ in this work. From this view, the lensed interface is the most optimal option for Fiber-to-Chip coupling. But coupling ability is not the exclusive aspect for the practical application. The fabrication cost must be considered. The method of using simple tapered interface or using another guide material is easier for the fabrication than the lensed interface. Thus the simple tapered interface is a more economical solution. \\
       
% the extensions of this work.
There are more and more interesting designs for the effective Fiber-to-Chip coupling. The tapered plasmonic waveguide in \cite{tapered_plasmonic_waveguides} is the application of SPP mode wave provided by metal/dielectric interface. This design may achieveed coupling efficiency over $100\%$. Alonso-Ramos involve grating as coupler in \cite{fiber_to_chip_grating_waveguides}  to extract beams into another planner waveguide. In his developments he reached the coupling efficiency of $65.6\%$. Besides above two designs section \ref{sect:optim_tapered_ext} and section \ref{sect:optim_lensed_ext} of this work also mention two extension for further development. The hybrid tapered interface and lensed interface with a neck are properly uneasy for fabrication, but as a simulation project they can still engage our attentions.
