%bewertung_sprachmodells
Vorher haben wir die Modellierung vorgestellt. Aber wie gut ist das entworfene Sprachmodell? Ist das aus unseren Trainingskorpora trainierte Modell auch g\"ultig bei einem unabh\"angigen Textkorpus? In der Informationstheorie wird dazu h\"aufig die \emph{Entropie}  eingef\"uhrt.
\\
\\
\emph{Entropie} ist ein Ma\ss \space f\"ur den mittleren Informationsgehalt oder auch die Informationsdichte eines Zeichensystems \cite{int_entropie}. Man kann durch die Entropietheorie die Informationsgehalte messen und bewerten, ob ein m-Gramm an eine Sprache angepasst werden kann. Man kann ebenfalls die Vorhersagekraft eines Sprachmodells messen,usw..\cite{book_speech}.
Die Entropie wird zun\"achst durch folgende Gleichung definiert:
%2.26
\begin{equation}
\label{equation:bewertung_01}
H(x)=-\sum_{x\in\chi}P(x)log_{b}(P(x))
\end{equation}
\\
\\
$x$ ist ein zuf\"alliges Ereignis. $\chi$ ist die Menge aller m\"oglichen Ereignisse. $-log_{b}(P(x))$ bedeute den Informationsgehalt des Ereignisses $x$, Basis $b$ ist h\"aufig gleich 2. Die Einheit der Entropie $H(x)$ ist \emph{bits}, wenn $b=2$.
Der Begriff "`Perplexit\"at"' wird auch in vielen F\"allen gebraucht. Perplexit\"at hat genau ein Zusammenhang (2.27) mit der Entropie. Perplexit\"at gibt das Ma\ss \space f\"ur die St\"arke der Einschr\"ankung durch das Sprachmodell an; also die mittlere Zahl der Wahlm\"oglichkeiten f\"ur das n\"achste Wort.
%2.27
\begin{equation}
\label{equation:bewertung_02}
PP(x)=2^{H(x)}=2^{-\sum_{x\in\chi}P(x)log_{b}(P(x))}
\end{equation}
\\
\\
Es folgt ein Beispiel \"uber die Anwendung der Entropie, die die Bandbreite zweier Systeme X und Y vergleicht. System $X$ enth\"alt Ereignisse $x1\thicksim x8$, System $Y$ enthalt Ereignisse $y1\thicksim y8$. Die folgende Tablle stellt die 64 Samples aus jedem Systeme dar. Und die Ereignisse sind wie in der Tabelle gezeigt kodiert.
%table 2.2
\begin{table}[h]
  \begin{center}
    \begin{tabular}{lccrlccr}
      \toprule
      \bf $X$ & \bf H\"aufig & \bf Wahrscheinlich & \bf Kode 
   	& \bf $Y$ & \bf H\"aufig & \bf Wahrscheinlich & \bf Kode\\    
      \midrule     
      $x1$ 		&  $32$ 		 	& $1/2$  							& $0$	
    & $y1$		&  $8$ 		 		& $1/8$  							& $001$	\\
      $x2$ 		&  $16$ 			& $1/4$  							& $10$
    & $y2$		&  $8$ 		 		& $1/8$  							& $010$	\\
     	$x3$ 		&  $8$ 		 		& $1/8$  							& $110$	
    & $y3$		&  $8$ 		 		& $1/8$  							& $011$	\\
      $x4$ 		&  $4$ 				& $1/16$  						& $1110$
    & $y4$		&  $8$ 		 		& $1/8$  							& $100$	\\
     	$x5$ 		&  $1$ 		 		& $1/64$   						& $111100$	
    & $y5$		&  $8$ 		 		& $1/8$  							& $101$	\\
      $x6$ 		&  $1$ 				& $1/64$  						& $111101$
   	& $y6$		&  $8$ 		 		& $1/8$  							& $110$	\\
     	$x7$ 		&  $1$ 		 		& $1/64$  						& $111110$	
    & $y7$		&  $8$ 		 		& $1/8$  							& $111$	\\
      $x8$ 		&  $1$ 				& $1/64$  						& $111111$
    & $y8$		&  $8$ 		 		& $1/8$  							& $000$	\\     
      \bottomrule
    \end{tabular}
  \end{center}
\caption{Samples aus System X und Y}
\label{tab:table_2}
\end{table}
\\
\\
Man kann genau berechnen, wie viele Bits bei der \"Ubertragung aller 64 Samples gebraucht wird. F\"ur Sample X wird 1*32+2*16+8*3+4*4+6*1*4 = 128 bits ben\"otigt. F\"ur Sample Y wird 3*64 = 192 bits ben\"otigt. D.h. das System Y braucht mehr Bandbreite als das System X. Weiterhin berechen wir auch die Entropie f\"ur beide Systeme.
%2.28

%\begin{multline}
\begin{align}
H(X) &=-\sum_{x\in\chi}P(x)log_{2}(P(x))\nonumber \\
		 &=-(1/2*log_{2}(1/2))+1/4*log_{2}(1/4)+1/8*log_{2}(1/8)+1/16*log_{2}(1/16)\nonumber \\
		 &+4*1/64*log_{2}(1/64))=2
\end{align}
%\end{multline}
%2.29
\begin{align}
H(Y)&=-\sum_{y\in\chi}P(y)log_{2}(P(y))\\
	  &=-(8*1/8*log_{2}(1/8)=3
\end{align}
\\
\\
Wie man erkannt, dass $H(Y)/H(X)=3/2=192/128$ ist . In diesem Beispiel kann man sofort erkennen, dass die Entropie als die Metrik der Bandbreitenanfordrung eines Systems bezeichnet wird.
\\
\\   
Vorher haben wir nur \"uber die Entropie eines Zufallsprozesses,der nur ein einzelnes Symbol einsch\"atzt, gesprochen. Man braucht tats\"achlich die Entropie f\"ur Symbolfolge.
%2.30
\begin{align}
H(w_{1},w_{2}...w_{n})&=H(w_{1}^{n})\nonumber \\
											&=-\sum{P(w_{1}^{n})log_{2}P(w_{1}^{n})}
\end{align}
\textbf{Entropierate}
%2.31
\begin{align}
H(L)&=\frac{1}{n}H(w_{1}^{n})\nonumber \\
		&=-\frac{1}{n}\sum_{w\in L}P(w_{1}^{n})log_{2}P(w_{1}^{n})
\end{align}
\\
\\
Wie das tats\"achliche Sprachmodell in der Gleichung(2.1) zeigt, kann $n$ unendlich gro\ss \space sein.
%2.32
\begin{equation}
H(L)=\lim_{n\to\infty}\frac{1}{n}H(w_{1}^{n})=\lim_{n\to\infty}-\frac{1}{n}\sum_{w\in L}P(w_{1}^{n})log_{2}P(w_{1}^{n})
\end{equation}
\\
\\
Nach dem "`Shannon-McMillan-Breiman theorem"' wird die Gleichung vereinfacht.
%2.33
\begin{equation}
H(L)=\lim_{n\to\infty}-\frac{1}{n}\sum_{w\in L}log_{2}P(w_{1}^{n})
\end{equation}
\\
\\
\textbf{Kreuzentropie}
$p(w_{1}^{n})$ in der Gleichung (2.31) ist die wahre Wahrscheinlichkeitsverteilung des Sprachmodells. Aber $p(w_{1}^{n})$ liegt nicht vor und soll gerade durch das M-Gramm gesch\"atzt werden. Man kann aus den Trainingsdatenbasen die Wahrscheinlichkeit $m(w_{1}^{n})$ sch\"atzen. Aus diesem Grund  wird hier der Kreuzentropie diskutiert.
%2.34
\begin{equation}
H(p,m)=\lim_{n\to\infty}-\frac{1}{n}\sum_{w\in L}p(w_{1}^{n})log_{2}m(w_{1}^{n})
\end{equation}
\\
\\
$p(w_{1}^{n})$ ist zwar nicht bekannt, aber nach dem "`Shannon-McMillan-Breiman Theorem"' wird die Gleichung vereinfacht. 
%2.35
\begin{equation}
H(p,m)=\lim_{n\to\infty}-\frac{1}{n}\sum_{w\in L}log_{2}m(w_{1}^{n})
\end{equation}
\\
\\
F\"ur jedes Modell $m$ gilt: \\
%2.36
\begin{equation}
H(p)\leq H(p,m)
\end{equation}
\\
\\
Und nur wenn $m=p$, gilt $H(p)=H(p,m)$. Mit anderen Worten, je kleiner $H(p,m)$ ist, desto genauer wird die Wahrscheinlichkeit $m$ an die wahre Wahrscheinlichkeit $p$ angen\"ahert.
In unseren folgenden Experimenten wird die Entropie sowie die Perplexit\"at dazu verwendet zu messen, wie gut das trainierte M-Gramm den eingegebenen Testkorpus modellieren kann.

