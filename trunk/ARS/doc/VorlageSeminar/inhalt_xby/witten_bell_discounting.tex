%witten_bell_discounting
Ein besserer und komplexerer Algorithmus lautet Witten-Bell-Discounting.
Dieser Algorithmus basiert auf dem Begriff "Things seen once". Man sch\"atzt die Wahrscheinlichkeit des ungesehenen M-Gramms aus der Anzahl der einmalig gesehenen M-Gramme. Zun\"achst sorgt man f\"ur die Wahrscheinlichkeitsmasse , die zu ungesehenen M-Grammen zugeordnet werden sollen. Die Gleichung (2.16) berechnet die  gesamte Wahrscheinlichkeit f\"ur  die M-Gramme mit Zero-Wahrscheinlichkeit.
%2.16
\begin{equation}
\label{equation:witten_bell_01}
\sum_{i:c_{i}=0}p_{i}^{*}(w_{i}|w_{n-m+1}^{n-1})=\frac{T(w_{n-m+1}^{n-1})}{N(w_{n-m+1}^{n-1})+T(w_{n-m+1}^{n-1})}
\end{equation}
\begin{itemize}
	\item $N(w_{n-m+1}^{n-1})$: Anzahl der Tokens\\
	\item $T(w_{n-m+1}^{n-1})$:Anzahl der einmalig gesehenen M-Gramms oder Typen
\end{itemize}
Um die genaue Wahrscheinlichkeit $p_{i}^{*}(w_{i}|w_{n-m+1}^{n-1})$ eines ungesehenen M-Gramms zu bestimmen, muss die Gleichung(2.16) auf beiden Seiten durch die Anzahl der ungesehenen M-Gramme dividiert werden:

%2.17
\begin{equation}
\label{equation:witten_bell_02}
p_{i}^{*}(w_{i}|w_{n-m+1}^{n-1})=\frac{T(w_{n-m+1}^{n-1})}{Z(w_{n-m+1}^{n-1})(N(w_{n-m+1}^{n-1})+T(w_{n-m+1}^{n-1}))}
\end{equation}
\begin{itemize}
	\item $Z(w_{n-m+1}^{n-1})=\sum_{i:c_{i}=0}1$: Anzahl aller nie gesehenen M-Gramme mit Vergangenheit $(w_{n-m+1}^{n-1})$
\end{itemize}
Jedes Null-Wahrscheinlichkeits M-Gramm erh\"alt einen gleichen Anteil an der Wahrscheinlichkeitsmasse (Division der Wahrscheinlichkeitsma\ss durch $Z(w_{n-m+1}^{n-1})$).
Die Wahrscheinlichkeit f\"ur gesehene M-Gramme \"andert sich entsprechend; dabei muss gelten:\\ 
$\sum_{w_{n}}p(w_{n}|w_{n-m+1}^{n-1})=1$.\\
Dann wird der ML-Sch\"atzwert  f\"ur ein gesehenes m-Gramm von (2.8) zu einer neuen Form (2.18). 

%2.18
\begin{equation}
\label{equation:witten_bell_03}
p^{*}(w_{n}|w_{n-m+1}^{n-1})=\frac{c(w_{n}|w_{n-m+1}^{n-1})}{N(w_{n-m+1}^{n-1})+T(w_{n-m+1}^{n-1})}
\end{equation}
\\
\\
In der allgemeinen Darstellung ergibt sich f\"ur ein M-Gramm  beim Witten-Bell-Discounting wie folgende Gleichung (2.19).

%2.19
\begin{equation}
\label{equationo:witten_bell_04}
p^{*}(w_{n}|w_{n-m+1}^{n-1})=\begin{cases}
\frac{c(w_{n},w_{n-m+1}^{n-1})}{N(w_{n-m+1}^{n-1})+T(w_{n-m+1}^{n-1})} & c(w_{n},w_{n-m+1}^{n-1})>0 \\
\frac{T(w_{n-m+1}^{n-1})}{Z(w_{n-m+1}^{n-1})(N(w_{n-m+1}^{n-1})+T(w_{n-m+1}^{n-1}))} & c(w_{n},w_{n-m+1}^{n-1})=0 
\end{cases}
\end{equation}
