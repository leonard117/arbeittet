%witten_bell_discounting
Ein besserer und komplexerer Algorithmus lautet Witen-Bell-Discounting.
Dieser Algorithmus basiert auf dem Begriff "Things see once". Man Sch\"atzt die Wahrscheinlichkeit des ungesehenen m-Gramms mit der Anzahl des erst mal gesehenen M-Gramms. Zun\"achst sorgt man f\"ur die Wahrscheinlichkeitsma\ss , die zu ungesehenen m-Gramm zuordnen sollen. Die Gleichung (2.16) berechnet die  gesamte Wahrscheinlichkeit f\"ur  die M-Gramme mit Zero-Wahrscheinlichkeit.
%2.16
\begin{equation}
\label{equation:witten_bell_01}
\sum_{i:c_{i}=0}p_{i}^{*}(w_{i}|w_{n-m+1}^{n-1})=\frac{T(w_{n-m+1}^{n-1})}{N(w_{n-m+1}^{n-1})+T(w_{n-m+1}^{n-1})}
\end{equation}

$N(w_{n-m+1}^{n-1})$: Anzahl der Tokens
$T(w_{n-m+1}^{n-1})$:Anzahl der erst mal gesehenen m-Gramms oder Typen

Um die genau Wahrscheinlichkeit $p_{i}^{*}(w_{i}|w_{n-m+1}^{n-1})$ eines ungesehenen M-Gramm zu bestimmen, muss die Gleichung(2.16) in beide Seit durch Anzahl der ungesehenen m-Gramms wie Gleichung (2.17).

%2.17
\begin{equation}
\label{equation:witten_bell_02}
p_{i}^{*}(w_{i}|w_{n-m+1}^{n-1})=\frac{T(w_{n-m+1}^{n-1})}{Z(w_{n-m+1}^{n-1})(N(w_{n-m+1}^{n-1})+T(w_{n-m+1}^{n-1}))}
\end{equation}

$Z(w_{n-m+1}^{n-1})=\sum_{i:c_{i}=0}1$:: Anzahl aller nie gesehenen m-Gramme mit Vergangenheit $(w_{n-m+1}^{n-1})$

Jedes Null-Wahrscheinlichkeit m-Gramm erh\"alt gleichen Anteil an der Wahrscheinlichkeitsma\ss (Division der Wahrscheinlichkeitsma\ss durch $Z(w_{n-m+1}^{n-1})$.
Die Wahrscheinlichkeit f\"ur ein gesehenen m-Gramme \"andert sich entsprechend; dabei muss gelten:\\ 
$\sum_{w_{n}}p(w_{n}|w_{n-m+1}^{n-1})=1$. Dann wird die ML-Sch\"atzwert  f\"ur ein gesehenes m-Gramm von (2.8) zu einer neuen Form (2.18). 

%2.18
\begin{equation}
\label{equation:witten_bell_03}
p^{*}(w_{n}|w_{n-m+1}^{n-1})=\frac{c(w_{n}|w_{n-m+1}^{n-1})}{N(w_{n-m+1}^{n-1})+T(w_{n-m+1}^{n-1})}
\end{equation}

Bei Zusammenfassung ist die Darstellung der Witten-Bell-Discounting f\"ur m-Gramm wie folgende Gleichung (2.19).

%2.19
\begin{equation}
\label{equationo:witten_bell_04}
p^{*}(w_{n}|w_{n-m+1}^{n-1})=\begin{cases}
\frac{c(w_{n},w_{n-m+1}^{n-1})}{N(w_{n-m+1}^{n-1})+T(w_{n-m+1}^{n-1})} & c(w_{n},w_{n-m+1}^{n-1})>0 \\
\frac{T(w_{n-m+1}^{n-1})}{Z(w_{n-m+1}^{n-1})(N(w_{n-m+1}^{n-1})+T(w_{n-m+1}^{n-1}))} & c(w_{n},w_{n-m+1}^{n-1})=0 
\end{cases}
\end{equation}
