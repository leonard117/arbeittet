%auto_ausfuehrung

Um das Toolkit automatisch auszuf\"uhren, wurde ein Perl-Script Programm entwickelt. Die Aufgabe des Trainings kann in die folgenden  3 Schritte eingeteilt werden.
\begin{itemize}
	\item Entpackung der Trainingskorpora
	\item Umstellung der Trainingskorpora
	\item Ausf\"uhrung des Toolkits
\end{itemize}
Dazu enth\"alt dieses Programme folgende f\"unf Dateien "`unpack.pl"', "`convertfile.pl"', "`call\_toolkitEx.pl"', "`Pfileoperation.pm"',  "`call\_toolkit.ini"'.\\
Die Datei "`unpack.pl"' hat die Aufgabe, die originale, im Format "`.z"' komprimierte  wsj Dateien zu extrahieren.\\
Die Datei "`convertfile.pl"': tauscht die ung\"ultigen Kuse (<pxxx></p><sxxx></s>) durch g\"ultigen Kuse (<s></s>) aus und ver\"andert alle Buchstaben in Kleine. \\
Die Datei "`call\_toolkitEx.pl"' ist das Hauptprogramm, das SLM-Toolkit aufzurufen.\\ 
Die Datei "`Pfileoperation.pm"' ist ein Perl-Modul\cite{book_perl} als gemeinsame Operationsbibliothek.\\
In der Datei "`call\_toolkit.ini"'  wird die Parameterkonfiguration dargestellt. Darin enthalten sind alle ben\"otigten Parameter wie die Pfade des Toolkits, der Trainingskorpoa, Testkorpus. Ebenso sind die Testparameter wie M-Gramm-Typ und Discounting-Typ auch in dieser Datei definiert. Au\ss erdem wird das "`\#"`  als Kommentarzeichen verwendet. \"Anderungen an dieser Konfigurationsdatei werden als unterschiedliche Versuche ausgef\"uhrt. Folgend ist ein Beispielkonfiguration angegeben:

\begin{lstlisting}
#notice: only the first value valid.
#==========================================================
#configuration for unpack und converfile
#unpack
#path of .z file 
zsource_path=..//data//source
#specificy unpack path
unpack_path=..//data//target
#converfile
#the output file information. the final file fullname will be
#$convertfile_path//output//$convertfile_name
convertfile_path=..//data//target
convertfile_name=output.text

#==========================================================
# configuration for call_toolkitEx
#SLM pfad
programsource=..//CMU/bin

#trainingcorpus fullpath
filename=...//data/target//output/output.text
#filename=..//data/target//output/output_872.text
#filename=..//data/target//output/output_88.text
#filename=..//data/target//output/output_89.text
#filename=..//data/target//output/output_full.text

#working path
filepath=..//data/target//output/

#cuse file
fileCcs=.//other//lm.ccs

#testcorpus
fileTestText=..//data//test//pp_et_05.nvp

#SLM toolkit running  parameter
#ngram=1
#ngram=2
ngram=3
#discounting=absolute
discounting=good_turing
#discounting=witten_bell
#discounting=linear
#=========================================================

\end{lstlisting}
