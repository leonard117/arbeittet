%auto_ausfuehrung

Um das Toolkit automatisch auszuf\"uhren, habe ich ein Perl-Script Programm entwickelt. 
Dieses Programme enth\"alten folgende f\"unf Dateien "`unpack.pl, convertfile.pl, Pfileoperation.pm, call\_toolkitEx.pl, call\_toolkit.ini"'.
"`unpack.pl"' hat die Aufgabe, die originale, im Format "`.z"' komprimierte  wsj Dateien zu extrahieren.
"`convertfile.pl"': tauscht die ung\"ultiger Kuse (<pxxx></p><sxxx></s>) mit g\"ultige Kuse (<s></s>) aus und ver\"andert alle Buchstaben in Kleine. 
"`call\_toolkitEx.pl"' ist das Hauptprogramm, das SLM-Toolkit aufzurufen. 
"`Pfileoperation.pm"' ist eine gemeinsame Operationsbibliothek
"`call\_toolkit.ini"' ist die Parameterkonfiguration. Darin enthalten Toolkits Verzeichnis, Trainingskorpus, Testkorpus, M-Grammtyp und Discountingtyp. Au\ss erdem wird das "`\# "`  als Kommentarzeichen verwendet. \"Anderungen an dieser Konfigurationsdatei werden als unterschiedliche Versuche ausgef\"uhrt. Folgende ist das Sample aktueller Datei. Den kompletten findet man unter \url{http://code.google.com/p/arsausarbeit/source/browse/#svn/trunk/arsausarbeit/perl_program/linux/script}\\

\begin{lstlisting}
#SLM pfad 
programsource=..//CMU/bin 
#trainingcurps 
filename=..//data/target//output/output.text 
filepath=..//data/target//output/ 
fileCcs=..//data//other//lm.ccs 
#testcurps 
fileTestText=..//data//test//pp_et_05.nvp 
#running  parameter 
ngram=1 
#ngram=2 
#ngram=3 
#discounting=absolute 
#discounting=good_turing 
discounting=witten_bell 

\end{lstlisting}
