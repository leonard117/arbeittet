\chapter{Kurze Einf�hrung}
\label{chapter:introduction}

In diesem Kapitel werden die wichtigsten LaTeX-Konstrukte kurz vorgestellt.\\
Das PDF-Dokument PDF \textit{short-math-guide.pdf} in dem ZIP-Ordner stellt eine gute Anlaufstelle bei Fragen und Problemen (siehe Ordner: \textsc{lesenswertes}).

\section{Formeln}
\label{section:equations}

Dieser Abschnitt befa�t sich mit der Verwendung von Formeln.

\subsection{Inline-Formeln}
\label{subsection:inlineEquation}

Inline-Formeln beginnen und enden mit dem \$-Zeichen: $e^x = \sum\limits_{n=0}^{\infty}\frac{x^n}{n!}$

\subsection{Die equation-Umgebung}
\label{subsection:defaultEquation}

Die im vorherigen Abschnitt \ref{subsection:inlineEquation} eingef�hrte Formel schreibt sich mit Hilfe der \textit{equation}-Umgebung wie folgt:
\begin{equation}
\label{equation:expFunktion}
 e^x = \sum\limits_{n=0}^{\infty}\frac{x^n}{n!}
\end{equation}
Die Exponential-Funktion aus Gleichung \eqref{equation:expFunktion} wird in der Mustererkennung h�ufig verwendet.
ACHTUNG: \textit{eqref} fa�t die Nummer der Gleichung automatisch in runde Klammern.

\section{Literaturquellen referenzieren}
\label{section:literaturref}

Die Literaturquellen werden mit Hilfe von \textsc{bibtex} referenziert. Die Quellen k�nnen in unterschiedlichen  ''.bib'' Dateien beschrieben sein (siehe z.B. \url{bibs/RefA}). Der Aufruf im Dokument erfolgt dann gem��:\\
\textbackslash bibliographystyle\{IEEEtran\}\\
\textbackslash bibliography\{IEEEfull,bibs/RefA, bibs/RefB\}\\
Eine Auflistung aller bibliographystyles findet man z.B. unter \url{http://www.cs.stir.ac.uk/~kjt/software/latex/showbst.html}. Hier wurde der \textit{IEEEtrans} style verwendet, der in der \url{bibs/IEEEtran.bst} definiert wurde. Die Abk�rzungen stehen in \url{bibs/IEEEfull.bib}, welche vor allen Bibliotheken eingebunden wurde. Alle referenzierten  Quellen werden automatisch in das Literaturverzeichnis eingf�gt, ein Fehler erscheint, wenn der \\cite\{\}-Befehl nicht verwendet wurde.\\
Eine Quelle aus einer Datenbank wird so referenziert: \textbackslash cite\{bibtexkey\}.\\
So ist \cite{Digalakis1993} z.B. aus Datenbank RefA und \cite{Lee2007} z.B. aus Datenbank RefB.

\section{Tabellen}
\label{section:tabular}

Dieser Abschnitt befa�t sich mit der Verwendung von Tabellen und den neuen Befehlen des \textit{booktabs}-Packets.
\begin{table}[h]
  \begin{center}
    \begin{tabular}{clr}
      \toprule
      \bf laufende & \bf Wert 1 & \bf Wert 2\\
      \bf Nummer  & [Einheit]  & [Einheit] \\
      \midrule
      $1$ &  $1\,000$ & $20$  \\
      $2$ &  $1\,500$ & $40$  \\
      \bottomrule
    \end{tabular}
  \end{center}
\caption{Eine einfache Tabelle}
\label{tab:table_1}
\end{table}

\section{Grafiken}
\label{section:figures}

Dieser Abschnitt stellt kurz das Einf�gen von Grafiken und das Ersetzen von Text innerhalb einer eps-Datei durch LaTeX-eigene Schriften und insbesonder auch Formeln vor. 

Die Verwendung des \textit{lesenswertes}-Packets wird in der \textit{psfrag-guide.pdf} Datei n�her erkl�rt (siehe Ordner: \textsc{usefullreadings}).\\
\\
ACHTUNG: Die Ersetzungen sind erst in der PS/PDF-Datei an der gew�nschten Stelle. In der DVI-Datei sind die Ersetzungen nur aufgelistet.
\begin{figure}[h]
  \begin{center}
    \resizebox{\textwidth}{!}
    {
    \psfrag{Hintergrund}[][c]{Hintergrund}
    \psfrag{Gesicht}[][c]{Gesicht}
    \psfrag{Detektionskaskade}[][ct]{Detektionskaskade}
    \psfrag{Analyse-}[][ct]{Analyse-}
    \psfrag{fenster}[][c]{fenster}
    \psfrag{W}[][cb]{$\mathcal{W}$}
    \psfrag{Stufe1:}[][ct]{Stufe 1:}
    \psfrag{Stufe 2:}[][ct]{Stufe 2:}
    \psfrag{Stufe 3:}[][ct]{Stufe 3:}
    \psfrag{Stufe 4:}[][ct]{Stufe 4:}
    \psfrag{20 Pixel}[][c]{$|\mathcal{W}_1'|$=$20$ Pixel} 
    \psfrag{40 Pixel}[][c]{$|\mathcal{W}_2'|$=$40$ Pixel} 
    \psfrag{160 Pixel}[][c]{$|\mathcal{W}_3'|$=$160$ Pixel} 
    \psfrag{244 Pixel}[][c]{$|\mathcal{W}_4'|$=$244$ Pixel} 
    \psfrag{T1=}[][c]{$T_1=$}
    \psfrag{T2=}[][c]{$T_2=$}
    \psfrag{T3=}[][c]{$T_3=$}
    \psfrag{0.345}[][c]{$0.345$}
    \psfrag{0.375}[][c]{$0.375$}
    \psfrag{0.385}[][c]{$0.385$}
    \psfrag{T4}[][c]{$T_4$}
    \includegraphics{beispiel.eps}
    }
  \end{center}
  \caption{Eine Grafik mit ersetzten Schriften}
  \label{fig:figure_1}
\end{figure} 




