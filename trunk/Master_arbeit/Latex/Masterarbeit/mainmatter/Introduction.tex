%Introduction
%\chapter{Introduction}

%introduce
For the research of photonic waveguides it is normally in use to project lights from optical fibers to photonic waveguides (Fiber-to-Chip). In this case the source fiber is generally connected with the laser source because a laser is diffraction limited and highly concentrated. As the signal source optical fibers have usually a lager end face than that of waveguides and the direct coupling from fibers to waveguides cause a very low coupling efficiency. In order to handle this problem tapered lensed fibers (TLF), which are optical fibers with a lens on the end face, are used to replace normal fibers. In Fiber-to-Chip  this new end face of fibers will focus rays emitted from fibers, so that more light power can be coupled into waveguides. The purpose of this work is to analyze the coupling ability from TLF to photonic waveguide through simulations in CST MWS  (CST Studio suite 2010) and optimize the Fiber-to-Chip interface to achieve more effective coupling.\\
In this work the related basic knowledge for research and analysis will be firstly introduced to make some terms in this work clear for readers. Then the chapter\ref{chp:model} will address readers information about the technical detail of the experimental objects and the modeling procedure. After then we will simulate the unoptimized coupling arrangement and analyze the coupling behavior. In chapter\ref{chp:optim} we will divide the development about the effective coupling between TLF and the waveguide  into four parts. In the first part, we aim at the effect of displacing the waveguide on the coupling efficiency. In the second part we will simulate the unoptimized coupling configuration in oil environment. Then in the third and forth parts we will provide two important techniques of waveguide interface, tapered and lensed interface, for promoting the coupling ability respectively.\\  
