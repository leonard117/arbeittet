%introduce
For the research of photonic waveguides it is normally in use to project lights from optical fibers to photonic waveguides (Fiber-to-Chip). In this case the source fiber is generally connected with the laser source because a laser is diffraction limited and highly concentrated. As the signal source optical fibers have usually a larger end face than that of waveguides and the direct coupling from fibers to waveguides cause a very low coupling efficiency. In order to handle this problem one consideration is to improve the propagating properties of the output beam from the fiber. For this reason tapered lensed fibers (TLF) \cite{TLF_mode_transforming,TLF_analysis}, which are optical fibers with a tapered end and a lens on the end face, are used to replace normal fibers. In Fiber-to-Chip this new end face of fibers will focus rays emitted from fibers, so that more light power can be coupled into waveguides. Besides this thinking at the fiber side more activities can proceed at the middle propagating course and waveguide interfaces. In recent researches there are lots of discussions about the design of waveguide interfaces. Tapered interfaces \cite{design_fabrication_tapered_waveguide}, grating interfaces \cite{fiber_to_chip_grating_waveguides} and other structures can greatly affect Fiber-to-Chip ability. In this work we discuss mainly about activities at the middle propagating course and waveguide interfaces.\\ 

As optimal designs for Fiber-to-Chip coupling, varieties of complex structures lead to great difficulties. Numerical methods are more suitable for solving Fiber-to-Chip problems. The purpose of this work, based on a given experimental setup about the Fiber-to-Chip problem, is to analyze different coupling configurations from TLF to photonic waveguide through simulations in CST MWS (CST Microwave Studio\textregistered) and Matlab programs, so that optimal Fiber-to-Chip interfaces can be found for achieving more effective coupling.\\

In this work the related basic knowledge for research and analysis are first introduced to clearify some terms in Fiber-to-Chip problem. Then the chapter \ref{chp:model} address information about the technical detail of the experimental objects and the modeling procedure. After then we simulate the unoptimized coupling arrangement and analyze the coupling behavior. In chapter \ref{chp:optim} we divide the development about the effective coupling between TLF and the waveguide into five sections. In the first section, we will talk about the effect of displacing the waveguide on the coupling efficiency. In the second section we will simulate the unoptimized coupling configuration in oil environment. Then in next three sections we consider designs of waveguide interfaces. Firstly, effects of material composing for waveguides are discussed. Then we provide two important structures of waveguide interfaces, tapered structure and lensed structure, for promoting the coupling ability respectively. At the end of this we conclude the above simulation results and give suggestions for optimizing the experimental setup.\\ 
