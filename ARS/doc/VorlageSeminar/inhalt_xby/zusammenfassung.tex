%zusammenfassung
Diese Beschreibung erkl\"art, wie mit Hilfe eines approximativen Modells von M-Gramm ein statistisches Sprachmodell entsteht. Darin werden unterschiedliche Gl\"attungsverfahren eingesetzt, um das Sprachmodell optimiert aufzubauen. Weiter wird die Beurteilung des Sprachmodells mit Hilfe der Entropie sowie der Perplexit\"at vorgestellt.
Durch Versuche mit der Sprachmodellrealisierung des SLM-Toolkits hat man erkannt, dass das Trigramm besser als Bigramm und Unigramm sind. Das Good-Turing Discounting erzielt ebenso in den Experimenten eine bessere Leistung als das Witten-Bell Discounting. 
\\
\\
Unterschiedliche Kombinationen von Backing-off mit Discounting sind wichtige Faktoren.
