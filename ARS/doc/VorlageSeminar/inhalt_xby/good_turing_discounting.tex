%good_turing_discounting
Good-Turing ist ein noch komplexerer Discounting-Algorithmus und liefert uns eine andere M\"oglichkeit  das Zero-Wahrscheinlichkeits-Problem zu l\"osen. In Witten-Bell, muss man zun\"achst die Wahrscheinlichkeitsmasse f\"ur ungesehene M-Gramme sparen. Aber in diesem Algorithmus wird die gesuchte Wahrscheinlichkeitsmasse aus der Gleichung (2.8) direkt berechnet und die zu den ungesehenen M-Grammen geh\"orenden Parameter $c(w_{n},w_{n-m+1}^{n-1})$ durch die Anzahl $N_{c}$ der  $c$ mal im Training auftretenden M-Gramme berechnet. Um die Bedeutung $N_{c}$ zu erkl\"aren, nimmt man das Unigramm als Beispiel. $N_{0}$ ist die Anzahl der ungesehenen M-Gramme, $N_{1}$ist die Anzahl der nur einmalig gesehenen M-Gramme, $N_{2}$ bedeutet, es gibt  M-Gramme, die nur 2 mal auftreten usw. . Die mathematische Darstellung f\"ur $N_{c}$ sieht man in der Gleichung (2.20).
%2.20
\begin{equation}
N_{c}=\sum_{c(w_{n})=c}1
\end{equation}
\\
\\
Wie sch\"atzt man eine neue H\"aufigkeit eines M-Gramms inklusiv ungesehenen M-Gramms? Man muss zuerst f\"ur alle $c's$(MLE) die $N_{c}'s$ zahlen. Dann man muss auch wissen, zu welche $c$(MLE) das M-Gramm $(w_{n},w_{n-m+1}^{n-1})$  geh\"ort. Weiterhin werden die bekannten $c(w_{n},w_{n-m+1}^{n-1})$, $N_{c+1}$ und $N_{c}$ eingesetzt. Man bekommt eine neue H\"aufigkeit von einem M-Gramm. Diese H\"aufigkeit wird in (2.8) verwendet.

%2.21
\begin{equation}
c^{*}(w_{n},w_{n-m+1}^{n-1})=(c(w_{n},w_{n-m+1}^{n-1})+1)\frac{N_{c+1}}{N_{c}}
\end{equation}
\\
\\
Die folgende Tabelle(2.1)  aus [2] zeigt eine Anwendung des Good-Turing Dsicounting f\"ur das Bigramm.
%table 2.1
\begin{table}[h]
  \begin{center}
    \begin{tabular}{clr}
      \toprule
      \bf $c$(MLE) & \bf $N_{c}$ & \bf $c^{*}$(GT)\\      
      \midrule
      $0$ &  $74,671,100,000$ & $0.0000270$  \\
      $1$ &  $2,018,046$ 			& $0.446$  \\
      $2$ &  $449,721$ 				& $1.26$  \\
      $3$ &  $188,933$ 				& $2.24$  \\
      $4$ &  $105,668$ 				& $3.24$  \\
      $5$ &  $68,379$ 				& $4.22$  \\
      $6$ &  $48,190$ 				& $5.19$  \\
      $7$ &  $35,709$ 				& $6.21$  \\
      $8$ &  $27,710$ 				& $7.24$  \\
      $9$ &  $22,280$ 				& $8.25$  \\
      \bottomrule
    \end{tabular}
  \end{center}
\caption{Die H\"aufigkeit der H\"aufigkeit beim Bigramm und Good-Turing Discounting von >Church und Galemit aus 22 Millionen Worten berechnet.}
\label{tab:table_1}
\end{table}
\\
\\
In tats\"achlichen Anwendung muss man nicht bei jeder H\"aufigkeit den M-Gramm in Good-Turing einsetzen. Man gibt ein H\"aufigkeits-Schwellenwert $k$ an, wenn ein M-Gramm nicht h\"aufig auftritt, weniger als $k$, f\"uhrt man dieses Gl\"attungsverfahren aus, sonst verwendet man die originale H\"aufigkeit $c(w_{n},w_{n-m+1}^{n-1})$. Das f\"uhrt die Gleichung (2.22) herbei.

%2.22
\begin{equation}
\label{equationo:witten_bell_04}
c^{*}(w_{n},w_{n-m+1}^{n-1})=\begin{cases}
c(w_{n},w_{n-m+1}^{n-1}) & c(w_{n},w_{n-m+1}^{n-1})>k \\
\frac{(c(w_{n},w_{n-m+1}^{n-1})+1)\frac{(k+1)N_{k+1}}{N{1}}}{1-\frac{(k+1)N_{k+1}}{N{1}}} & 0 \leq c(w_{n},w_{n-m+1}^{n-1})<k 
\end{cases}
\end{equation}