%add_one_smoothing
Das Add-One Discounting liefert eine andere L\"osung des Zero-Wahrscheinlichkeits-Problems. Man berechnet den ML-Sch\"atzwert aus der Gleichung (2.8). In der Add-One-Methode addiert man Eins zu allen Zahlen der W\"orter (oder m-Gramm) vor dem Normalisierungsverfahren.  Die Gleichung (2.8) wird durch (2.14) ersetzt.

%2.14
\begin{equation}
\label{equation:add_one_01}
p^{*}(w_{n}|w_{n-m+1}^{n-1})=\frac{c(w_{n},w_{n-m+1}^{n-1})+1}{c(w_{n-m+1}^{n-1})+V}=\frac{c_{k}^{*}}{N+V}
\end{equation}
\\
\\
Hier ist $V$ die Vokabulargr\"o\ss e (oder die gesamte Anzahl der m-Gramm-Typen) , damit \\ 
$\sum_{w_{n}}p^{*}(w_{n}|w_{n-m+1}^{n-1})=1$
\\
\\
Nach diesem Verfahren wird die Wahrscheinlichkeit aller angesehenen W\"orter relativ geringer. Und die W\"orter mit Null-Wahrscheinlichkeit bekommen einen gleichen neuen Sch\"atzwert $\frac{1}{N+V}$. Es gilt:
%2.15
\begin{equation}
p^{*}(w_{n}|w_{n-m+1}^{n-1})=\begin{cases}
\frac{c(w_{n},w_{n-m+1}^{n-1})+1}{N+V} & c(w_{n},w_{n-m+1}^{n-1})>0 \\
\frac{1}{N+V} & c(w_{n},w_{n-m+1}^{n-1})=0 
\end{cases}
\end{equation}
\\
\\
Add-One-Discounting ist ein sehr einfacher Algorithmus und funktioniert nicht so gut, weil die Wahrscheinlichkeitsmasse der "`Non-Zero-M-Gramme"' nach diesem Verfahren sehr viel ge\"andert wird\cite{book_speech}. Folglich wird Add-One-Discounting  praktisch kaum verwendet.