%Einleitung
Bei einem Spracherkkenner ist die Sprachmodellierung sehr schwer. Um ein Sprachmodell f\"ur eine bestimmte Sprache zu entwerfen, muss man aus dem Trainingskorpus die bedingte Wahrscheinlichkeit $p(w_{n}|w_{1},..w_{n-1})$   vom Sprachmodell $p(w_{1}^n)$   bestimmen.
\\
\\
Die vorliegende Hausarbeit versucht zu erkl\"aren, wie ein realisierbares Modell durch Training erzeugt wird und die L\"osung des Problems der eingeschr\"ankten Trainingdatenbanken erreicht wird. 
\\
\\
Um diese zu erreichen, werden in diesem Artikel zuerst ein approximatives Modell, M-Gramm, und unterschiedliche Discounting-Algorithmuen f\"ur das 0-Wahrscheinlichkeits-problem vorgestellt. Am Ende werden unsere Ergebnisse durch ein SLM-Toolkit angezeigt.
